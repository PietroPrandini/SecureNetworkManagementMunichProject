% Sample of a report for an attack as requested at wednesday, 5 December 2018, mail "Assignemnt Lab 19"
\subsection{Attack Name}
% SCAPY program
\subsubsection{SCAPY program}
\lstinputlisting{scapy/DropCommunication.py}
Sending this type of messages we are telling to the victims that the other end of its current transmission (ftp for example) is no longer available.\\
But how is made this type of ICMP packet? From the \href{https://www.iana.org/assignments/icmp-parameters/icmp-parameters.xhtml}{documentation of \textit{iana}} we can understand that type 3 stands for “Destination Unreachable” and with code 1 we specify “Host Unreachable”.\\
Other code for “Destination Unreachable” could be:\par
\medskip
\begin{tabular}{|c|l|}
  \hline
  \textbf{code} & \textbf{description} \\
  \hline
  0 & Net Unreachable \\
  1 & Host Unreachable \\
  2 & Protocol Unreachable \\
  3 & Port Unreachable \\
  4 & Fragmentation Needed and Don't Fragment was Set \\
  5 & Source Route Failed \\
  6 & Destination Network Unknown \\
  7 & Destination Host Unknown \\
  8 & Source Host Isolated \\
  9 & Communication with Destination Network is Administratively Prohibited \\
  10 & Communication with Destination Host is Administratively Prohibited \\
  11 & Network Unreachable for Type of Service \\
  12 & Host Unreachable for Type of Service \\
  13 & Communication Administratively Prohibited \\
  14 & Host Precedence Violation \\
  15 & Precedence cutoff in effect \\
  \hline
\end{tabular}

% Wireshark presenting  the attacker's messages
\subsubsection{Attacker's messages and result}
Using the wireshark filter: \textit{icmp.type==3 \&\& icmp.code==1}, we get:\par
\medskip
\begin{tabular}{|l|l|l|l|l|l|}
  \hline
  \textbf{No.} & \textbf{Time} & \textbf{Source} & \textbf{Destination} & \textbf{Length} & \textbf{Info} \\
  \hline
  30 & 198.442781000 & 192.168.60.60 & 192.168.40.50 & ICMP & Destination unreachable (Host unreachable) \\
  33 & 198.843752000 & 192.168.60.60 & 192.168.40.50 & ICMP & Destination unreachable (Host unreachable) \\
  38 & 198.844320000 & 192.168.60.60 & 192.168.40.50 & ICMP & Destination unreachable (Host unreachable) \\
  39 & 210.969312000 & 192.168.40.50 & 192.168.60.60 & ICMP & Destination unreachable (Host unreachable) \\
  41 & 211.687367000 & 192.168.40.50 & 192.168.60.60 & ICMP & Destination unreachable (Host unreachable) \\
  43 & 212.459272000 & 192.168.40.50 & 192.168.60.60 & ICMP & Destination unreachable (Host unreachable) \\
\hline
\end{tabular}

% Method recommended to protect the Network against such an attack
\subsubsection{How to protect the network}
The best way to block this type of attack is to use a firewall and block all the ICMP packets. In this way the attacker can’t reach the victim because the denial ICMP packet could not even reach the victim.\par
