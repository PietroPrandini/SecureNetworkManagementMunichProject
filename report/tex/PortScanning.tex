% Sample of a report for an attack as requested at wednesday, 5 December 2018, mail "Assignemnt Lab 19"
\subsection{Port Scanning}
% SCAPY program
\subsubsection{SCAPY program}
\lstinputlisting{scapy/PortScanning.py}
The objective of this Scapy program is to be able to get a range of opened ports from a victim ip. This program aim the victim located at ip 192.168.40.50 and send a TCP packet with SYN flag for scanning the range of ports from 1 to 1024 with an interval of 5 milliseconds and get the response using the sr() function which sends and receives packets.\par

% Wireshark presenting  the attacker's messages
\subsubsection{Attacker’s messages and result}
%\includegraphics[width=16cm]{img/AttackNameWiresharkPacketsCaptured.png}
\begin{center}
\begin{longtable}{|l|l|l|l|l|l|l|l|p{2.5cm}|}
\hline
{\tiny{\textbf{No.}}} & {\tiny{\textbf{Time}}} & {\tiny{\textbf{Source}}} & {\tiny{\textbf{Destination}}} & {\tiny{\textbf{Protocol}}} & {\tiny{\textbf{Src Port}}} & {\tiny{\textbf{Dest Port}}} & {\tiny{\textbf{Length}}} & {\tiny{\textbf{Info}}} \\
\hline
\endhead
5 & 0.004414000 & 192.168.62.60 & 192.168.40.50 & TCP & ftp-data & 1 & 56 & ftp-data $>$ tcpmux [SYN] Seq=0 Win=8192 Len=0\\
\hline
10 & 0.005059000 & 192.168.62.60 & 192.168.40.50 & TCP & ftp-data & 1 & 62 & [TCP Out-Of-Order] ftp-data $>$ tcpmux [SYN] Seq=0 Win=8192 Len=0\\
\hline
11 & 0.005210000 & 192.168.40.50 & 192.168.62.60 & TCP & tcpmux & 20 & 62 & tcpmux $>$ ftp-data [RST, ACK] Seq=1 Ack=1 Win=0 Len=0\\
\hline
16 & 0.005575000 & 192.168.40.50 & 192.168.62.60 & TCP & tcpmux & 20 & 62 & tcpmux $>$ ftp-data [RST, ACK] Seq=1 Ack=1 Win=0 Len=0\\
\hline
17 & 0.011168000 & 192.168.62.60 & 192.168.40.50 & TCP & ftp-data & 2 & 56 & ftp-data $>$ compressnet [SYN] Seq=0 Win=8192 Len=0\\
\hline
22 & 0.011645000 & 192.168.62.60 & 192.168.40.50 & TCP & ftp-data & 2 & 62 & [TCP Out-Of-Order] ftp-data $>$ compressnet [SYN] Seq=0 Win=8192 Len=0\\
\hline
23 & 0.011765000 & 192.168.40.50 & 192.168.62.60 & TCP & compressnet & 20 & 62 & compressnet $>$ ftp-data [RST, ACK] Seq=1 Ack=1 Win=0 Len=0\\
\hline
28 & 0.012130000 & 192.168.40.50 & 192.168.62.60 & TCP & compressnet & 20 & 62 & compressnet $>$ ftp-data [RST, ACK] Seq=1 Ack=1 Win=0 Len=0 \\
\hline
\end{longtable}
\end{center}


This is an example of the scan for the first 2 ports which they respond with [RST,ACK+meaning that there is not service running on that port.
Scapy used the port 20 for sending the request as default but could be changed using:\\
\begin{lstlisting}
packet = IP(dst="192.168.40.50") / TCP(dport=(1,1024), sport=8888, flags="S")
\end{lstlisting}
for example but this is not what we are interested.\par
Using the following wireshark’s filter: \\
\begin{lstlisting}
tcp.flags.ack==1 && tcp.flags.syn==1
\end{lstlisting}
we can obtain the packets that respond to the tcp request with a [SYN, ACK] flag meaning that there is a service installed using that port:\par
\medskip
\begin{center}
\begin{longtable}{|l|l|l|l|l|l|l|l|p{2.5cm}|}
\hline
{\tiny{\textbf{No.}}} & {\tiny{\textbf{Time}}} & {\tiny{\textbf{Source}}} & {\tiny{\textbf{Destination}}} & {\tiny{\textbf{Protocol}}} & {\tiny{\textbf{Src Port}}} & {\tiny{\textbf{Dest Port}}} & {\tiny{\textbf{Length}}} & {\tiny{\textbf{Info}}} \\
\hline
\endhead
263 & 0.156556000 & 192.168.40.50 & 192.168.62.60 & TCP & ssh & 20 & 62 & ssh $>$ ftp-data [SYN, ACK] Seq=0 Ack=1 Win=29200 Len=0 MSS=1460 \\
\hline
268 & 0.157021000 & 192.168.40.50 & 192.168.62.60 & TCP & ssh & 20 & 62 & [TCP Out-Of-Order] ssh $>$ ftp-data [SYN, ACK] Seq=0 Ack=1 Win=29200 Len=0 MSS=1460 \\
\hline
965 & 0.567846000 & 192.168.40.50 & 192.168.62.60 & TCP & http & 20 & 62 & http $>$ ftp-data [SYN, ACK] Seq=0 Ack=1 Win=29200 Len=0 MSS=1460 \\
\hline
970 & 0.568307000 & 192.168.40.50 & 192.168.62.60 & TCP & http & 20 & 62 & [TCP Out-Of-Order] http $>$ ftp-data [SYN, ACK] Seq=0 Ack=1 Win=29200 Len=0 MSS=1460 \\
\hline
\end{longtable}
\end{center}


We can understand that in the victim’s pc is running a ssh and http server.\par

% Method recommended to protect the Network against such an attack
\subsubsection{How to protect the network}
A way to protect the network from this type of attack can be to use a firewall to block the packet request from an host that has already request 10 tcp messages in 1 second or a similar tcpMessages/Time ratio.\\
In this way we can use the network normally but when there is a particular activity in the network we have countermeasures to protect the clients.\par
